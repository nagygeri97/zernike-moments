Image moments and moment invariant features are widely used for image analysis and pattern recognition. The system of orthogonal Zernike polynomials (defined over the complex unit disk) proved to be a useful basis for series expansions because of certain invariance properties.

Conventionally, for multichannel color images, RGB decomposition or grayscale conversion was used. However, in the recent past, quaternion algebra has been employed to various conventional moments to analyze a color image holistically. \citeauthor{qfmm}~\cite{qfmm} introduced quaternion Fourier-Mellin moments (QFMMs) which are an extension of the conventional Fourier-Mellin moments for the grayscale image. They also proposed their invariants on rotation, scale, and translation for color object recognition. \citeauthor{qzmi}~\cite{qzmi} proposed the quaternion Zernike moments (QZMs), generally overperforming other similar approaches in these aspects, due to the natural invariances of Zernike functions. The same quaternion techniques were applied successfully to other function systems (e.g.~\cite{Shao, chebyshev-fourier}), yielding similar results.

In this thesis we introduce a method of transforming a digital RGB image inside the unit circle onto a points system providing discrete orthogonality. Using this points system for the discretization of the QZMIs, we have achieved significant improvements in the image recognition and reconstruction capabilities of the method, especially under noisy conditions.