\documentclass[bigger]{beamer}

\usepackage[utf8]{inputenc}
\usepackage[T1]{fontenc}
\usepackage{graphicx}
\usepackage{lmodern}
\usepackage{textpos}
\frenchspacing



% Fordításhoz: pdflatex



\usebackgroundtemplate{%
\includegraphics[width=\paperwidth,height=\paperheight]{background.jpg}%
}

%\setbeamercolor{title}{fg=white}
%\setbeamercolor{author}{fg=white}
%\setbeamercolor{institute}{fg=white}
%\setbeamercolor{date}{fg=white}
\setbeamercolor{frametitle}{fg=white}
%
%\title{\bf Presentation heading}
%\author{Sample Samuel}
%\institute{Eötvös Loránd University (ELTE), Budapest, Hungary}
%\date{2018}

\begin{document}

{
\usebackgroundtemplate{\includegraphics[width=\paperwidth]{title.jpg}}%
\begin{frame}

\color{white}{

\textbf{\Large{Prezentáció címe}}

\bigskip

\Large{Előadó neve}

\bigskip

\footnotesize{Eötvös Loránd Tudományegyetem, Informatikai Kar}
\smallskip

\footnotesize{2018. június 11.}
\vspace{2em}

% Szakmai beszámolókhoz: (Igazi neveket kérünk feltüntetni.) Más előadáson nem kell.
\begin{footnotesize}
Témavezető: Dr. Vezető Vilmos\\
Szoftvertechnológia, Rendszerek Alkotóműhely\\
Mesterséges Valós Idejű Érzékelés Munkacsoport\\
\end{footnotesize}
\bigskip

% pozícionáljuk a dia aljához közel
\begin{textblock}{8}(0,2.5)
\footnotesize{EFOP-3.6.3-VEKOP-16-2017-00001}
\end{textblock}
}


\end{frame}
}

\begin{frame}{Dia címe}
\begin{itemize}
\item 1
\item 2
\item 3
\end{itemize}
\end{frame}

% this slide need not be used in the presentation, but must be
% present when you archieve your talk

{
\usebackgroundtemplate{\includegraphics[width=\paperwidth]{title.jpg}}%
\begin{frame}{}

\textbf{\huge\color{white} KÖSZÖNÖM}

\bigskip

\textbf{\huge\color{white} A FIGYELMET!}

\end{frame}
}

\end{document}
