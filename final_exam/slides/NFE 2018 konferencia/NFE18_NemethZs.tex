\documentclass{beamer}

\usepackage[utf8]{inputenc}
\usepackage[T1]{fontenc}
\usepackage{graphicx}
\usepackage{lmodern}
\usepackage{textpos}
\frenchspacing



% Fordításhoz: pdflatex

\newtheorem{remark}{Remark}

\def\n{n\in\N}
\def\x{x\in\R}
\def\N{{\mathbb N}}
\def\R{{\mathbb R}}
\def\C{{\mathbb C}}
\def\Z{{\mathbb Z}}
\def\D{{\mathbb D}}
\def\B{{\mathfrak B}}
\def\T{{\mathbb T}}
\def\btheta{{\boldsymbol{\vartheta}}}
\def\bgamma{{\boldsymbol{\gamma}}}
\def\bl{{\mathbf{l}}}
\def\bk{{\mathbf{k}}}
\def\bt{{\mathbf{t}}}
\def\be{{\mathbf{e}}}

\DeclareMathOperator{\spannedby}{span}
\DeclareMathOperator{\e}{e}
\DeclareMathOperator{\soc}{soc}

\def\xkng{x_{k,M}}
\def\xkkng{x_{k,n}^{(\gamma)}}
\def\xkkn{x_{k,n}}
\def\xkN{x_{k,M}}
\def\xkm{x_{k,M}}
\def\tkNg{\theta_{k,M}^{(\gamma)}}

\def\ab{(\alpha,\beta)}
\def\wab{w_{\alpha,\beta}}
\def\wgd{w_{\gamma,\delta}}
\def\lkm{\lambda_{k,M}}

\def\la{\langle}

\def\ykng{y_{k,M}}

\newtheorem{defi}{Definition}
\newtheorem{define}{Definition}
\newtheorem{allitas}{Statement}
\newtheorem{alli}{Statement}
\newtheorem{kov}{Corollary}
\newtheorem{problema}{Problem}


\usebackgroundtemplate{%
\includegraphics[width=\paperwidth,height=\paperheight]{background.jpg}%
}

%\setbeamercolor{title}{fg=white}
%\setbeamercolor{author}{fg=white}
%\setbeamercolor{institute}{fg=white}
%\setbeamercolor{date}{fg=white}
\setbeamercolor{frametitle}{fg=white}
%
%\title{\bf Presentation heading}
%\author{Sample Samuel}
%\institute{Eötvös Loránd University (ELTE), Budapest, Hungary}
%\date{2018}

\begin{document}

{
\usebackgroundtemplate{\includegraphics[width=\paperwidth]{title.jpg}}%
\begin{frame}

\color{white}{

\textbf{\Large{Projection Properties of de la Vall\'ee Poussin type operators}}

\bigskip

\Large{Zsolt Németh}

\bigskip

\footnotesize{Dept. of Numerical Analysis, Eötvös Loránd University}
\smallskip

\footnotesize{31st August, 2018.}
\vspace{2em}

% Szakmai beszámolókhoz: (Igazi neveket kérünk feltüntetni.) Más előadáson nem kell.
%\begin{footnotesize}
%Témavezető: Dr. Vezető Vilmos\\
%Szoftvertechnológia, Rendszerek Alkotóműhely\\
%Mesterséges Valós Idejű Érzékelés Munkacsoport\\
%\end{footnotesize}
%\bigskip

% pozícionáljuk a dia aljához közel
\begin{textblock}{8}(0,2.5)
\footnotesize{EFOP-3.6.3-VEKOP-16-2017-00001}
\end{textblock}
}


\end{frame}
}

\frame{
\frametitle{Trigonometric Fourier series}

\vskip 10mm
Complex trigonometric system:
\[
\varepsilon_j(x):=e^{ijx}\qquad
(\x,\ j\in\Z)
\]

Trigonometric Fourier coefficients ($f\in L_1(0,2\pi)$ complex valued function):
\[
\hat f(j):=\frac{1}{2\pi}\int_{-\pi}^{\pi}f(t)e^{-ijt}dt\qquad (j\in\Z).
\]
Trigonometric Fourier series of $f$:
\[
S[f]:=\sum_{j\in\Z}\hat f(j)\varepsilon_j.
\]
The $n$-th partial sum:
\[
(S_nf)(x):=\sum_{j=-n}^n\hat f(j)\varepsilon_j(x)\qquad (\x).
\]
}

\frame{

\vskip 15mm
First we consider the Banach space $(C_{2\pi}, \| \cdot \|_\infty)$.

The set of trigonometric polynomials of degree at most $n\in \N$:
\[
\mathcal{T}_n:=\text{span}\; \{ \varepsilon_j\ :\ -n\leq j\leq n\; \}.
\]
Now $S_n:~C_{2\pi}\to \mathcal{T}_n$ is a (bounded) linear operator with the projection property
\[(S_n g)(x)=g(x) \qquad (g\in\mathcal{T}_n, x\in\R).\]
\begin{theorem}[Faber--Marcinkiewicz--Berman]
Let $T_n:~C_{2\pi}\to \mathcal{T}_n$ denote a linear (trigonometric) projection, i.e. suppose that $T_n g=g$, $(g\in\mathcal{T}_n)$. Then we have
\[
\|T_n\| \geq \| S_n \|.
\]
\end{theorem}

We remark that $\| S_n \| = \frac{4}{\pi^2}\log n + O(1)$.
}


\frame{
\frametitle{(Generalized) de la Vall\'ee Poussin means}
The de la Vall\'ee Poussin means of (trigonometric) Fourier series:
\[
V_{n,m}f:=\frac{1}{m+1} \sum_{k=0}^m S_{n+k}f.
\]
Special cases: 
\begin{enumerate}
  \item partial sum operator $S_n$ ($m=0$);
  \item Fej\'er means ($n=0$);
  \item classic de la Vall\'ee Poussin means ($n=m+1$).
\end{enumerate}

Now we have $V_{n,m}:~C_{2\pi}\to\mathcal{T}_{n+m}$ and $(V_{n,m}g)(x)=g(x)$, where $g\in\mathcal{T}_n,x\in\R$.

In this talk we deal with this type of projection property.
}

\frame{

\vskip 15mm
\begin{theorem}[Nikolaev]
Let $n,m\in\N$, $n\geq 1$ and let $T_{n,m}:~C_{2\pi}\to \mathcal{T}_{n+m}$ denote a de la Vall\'ee Poussin type projection, i.e. a linear operator for which $T_{n,m} g=g$, $(g\in\mathcal{T}_n)$. Then there exists a positive constant $c\in\R$, independent of $n,m$, such that
\[
\|T_{n,m}\| \geq c\log\frac{n+m}{m+1}.
\]
\end{theorem}

We remark that for de la Vall\'ee Poussin means, $$\|V_{n,m}\| = \frac{4}{\pi^2}\log \frac{n+m}{m+1}+O(1).$$

Question/problem: Is the norm of $V_{n,m}$ minimal among these projections?
}

\frame{

\vskip 15mm
\begin{theorem}[Bernstein; Deregowska,Lewandowska]
If $n=m+1$ and $n>1$, then we have $\|T_{n,n-1}\| \geq \|V_{n,n-1}\|$, i.e. the minimality of the classical DLVP means.
\end{theorem}

\begin{theorem}[Nikolaev]
If $n|m+1$ and $n>1$, then the corresponding DLVP operator $V_{n,m}$ is minimal.
\end{theorem}

\begin{theorem}[Bernstein]
If $m+1$ is even, then the corresponding DLVP operator $V_{n,m}$ is minimal.
\end{theorem}

\begin{theorem}[NZs]
Let $n,m\in\N$, $n\geq 1$ and let $T_{n,m}:~C_{2\pi}\to \mathcal{T}_{n+m}$ denote a de la Vall\'ee Poussin type projection. Then, $\|T_{n,m}\| \geq \|V_{n,m}\|$ holds if and only if $(m+1,2n+m+1)>1$.
\end{theorem}
}

\frame{
\frametitle{Ingredients}

\vskip 8mm
\begin{enumerate}
  \item Simple kernel with explicit roots
  
  The $n$-th partial sum can be expressed as
\[
 \left(S_n f\right)(x)=\left(D_n\ast f\right)(x)=\frac{1}{2\pi}\int_0^{2\pi} f(y) D_n(x-y)~dy,
\]
where
\[
 D_n(x)=\frac{\sin\frac{2n+1}{2}x}{\sin \frac{x}{2}}, \qquad (x\in\R).
\]

For the de la Vall\'ee Poussin means: $V_{n,m}=G_{n,m}\ast f$,
where
\[
G_{n,m}(x) = \frac{1}{m+1} \sum_{k=0}^m D_{n+k}(x) = \frac{\sin\frac{m+1}{2}x\ \sin\frac{2n+m+1}{2}x}{(m+1)\sin^2 \frac{x}{2}}.
\]
  \item Mimicking Cheney \emph{et al.}: $\|T_{n,m}\|=\frac{1}{2\pi} \|G_{n,m}+y\|_1$, where
  $y\in\mathcal{T}_{n+m}\setminus\mathcal{T}_{n}$.
\end{enumerate}
}

\frame{
\frametitle{Generalisations}
\vskip 8mm
\begin{theorem}[Lambert]
The Faber--Marcinkiewicz--Berman theorem remains true if we replace $C_{2\pi}$ with $L_1(0,2\pi)$.
\end{theorem}
This is a (relatively) simple observation.

Our main result also remains true for functions $f\in L_1(0,2\pi)$.

We also have the algebraic polynomial variants for partial sums of Chebyshev series for $f\in L_1(-1,1)$.

Question: What is a minimal projection in the other cases? How far the DLVP operators are from being minimal?

\begin{theorem}[NZs]
$\|V_{n,m}\| \leq \|T^*_{n,m}\|+O(1)$.
\end{theorem}
}

\frame{
\frametitle{Multivariate extension}

\vskip 8mm
Fix $d>1,~d\in\N$, and denote the $d$-dimensional torus by $\T^d=\R^d$ (mod $2\pi\Z^d$) ($\Z=\{0,\pm 1,\pm 2,\ldots\}$).

The Fourier series of $g\in C(\T^d)$:
\[
g(\btheta) \sim \sum_\bk \hat{g}(\bk)\e^{i\bk\cdot\btheta}, \qquad \hat{g}(\bk)=\frac{1}{(2\pi)^d} \int_{\T^d} g(\bt)\e^{-i\bk\cdot\bt} d\bt,
\]
where $\btheta=(\vartheta_1,\vartheta_2,\ldots,\vartheta_d)\in\T^d$, $\bk=(k_1,k_2,\ldots,k_d)\in\Z^d$ and $\bk\cdot\btheta=\sum_{l=1}^d k_l \vartheta_l$ (scalar product).

The $n$-th \emph{triangular} partial sum:
\[
S_{n,d} (g,\btheta) = \sum_{|\bk|_1 \leq n} \hat{g}(\bk)\e^{i\bk\cdot\btheta} \qquad (n\in\N_0=\{0,1,2,\ldots\}),
\]
where $|\bk|_1 = \sum_{l=1}^d |k_l|$ (the $l_1$ norm of multiindex $\bk$).
}

\frame{

\vskip 15mm
$d$-dimensional de la Vallée Poussin means ($n,m \in\N_0$):
\[
V_{n,m,d} (g,\btheta) = \frac{1}{m+1} \sum_{j=0}^m S_{n+j,d} (g,\btheta).
\]

Denote the set of trigonometric polynomials
\[
\sum_{|\bk|_1 \leq n}  a_\bk \e^{i\bk\cdot\btheta},
\]
by $\mathcal{T}_{n,d}$, where $\bk=(k_1,k_2,\ldots,k_d)\in \Z^d$.

\begin{theorem}
Fix $d\geq 1$, $n,m\in\N$, $n\geq 1$. Let $T_{n,m,d}:~C(\T^d) \rightarrow \mathcal{T}_{n+m,d}$ denote a de la Vall\'ee Poussin type linear projection, i.e. $T_{n,m,d}(g,\btheta) = g(\btheta)$, where $g\in\mathcal{T}_{n,d}$. Now
\[
\|T_{n,m,d}\| \geq c \left( \log \frac{n+m}{m+1} \right)^d,
\]
where $c>0$ independent of $n,m$.
\end{theorem}
}

\frame{
\frametitle{The End}
\begin{center}
\huge Happy Birthday to All Celebrants!
\end{center}
}

% this slide need not be used in the presentation, but must be
% present when you archieve your talk

{
\usebackgroundtemplate{\includegraphics[width=\paperwidth]{title.jpg}}%
\begin{frame}{}

\textbf{\huge\color{white} THANK YOU}

\bigskip

\textbf{\huge\color{white} FOR YOUR ATTENTION!}

\end{frame}
}

\end{document}
