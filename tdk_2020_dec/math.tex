In this chapter we describe our proposed approach to the discretization of continuous quaternion Zernike moments. Our aim is to ensure simple and reversible computability, but achieve considerable reduction in potential computational inaccuracies at the same time. For this, we define a system of sampling points $(r_k,\theta_j)$, over which the integral discretization \eqref{eq:QZMappr} maintains advantageous theoretical properties. 

We emphasize that for the system of Zernike functions, there is no known system of points providing a "natural" discretization and discrete orthogonality. The idea behind our approach is motivated by the work of Margit Pap and Ferenc Schipp \cite{schipp}, which is in fact the first example of a point system over which the discrete complex-valued Zernike functions are orthogonal. On the other hand, it is not "natural" in the sense that the number of nodes needed to achieve this is somewhat greater (in fact, almost twice as much) than what the order of functions would suggest.

Formula \eqref{RJacobi} helps us understand the reason why a natural discretization is so difficult to find for Zernike functions: even though the azimuthal components could be discretized easily, it is not the case for the radial terms which are shifted Jacobi polynomials of \emph{varying} paramters. 

Nonetheless, it is the same formula that provides an idea how to succeed: we combine the natural discretization of the azimuthal components with an orthogonality relation ensured by the Gaussian quadrature formula for the radial polynomials.

For this, we shall define the suitable quadrature nodes and coefficients as the roots of Legendre polynomials and the corresponding Chrisfottel numbers. Fix a positive integer $N$. Let us denote by $\rho_{k,N}$, $k=1,\ldots,N$ the roots of the N-th order Legendre polynomial \cite{Szego}. Notating the fundamental polynomials of Lagrange interpolation with respect to the roots $\rho_{k,N}$ as $\ell_{k,N}$, we define the constants
\[
	\mathcal{A}_{k,N} = \int_{-1}^{1} \ell_{k,N}(x)\ dx, \qquad (k=1,\ldots,N).
\]

Then, the system of sampling points $X_N$ over the unit disk $\D$ is defined in polar form by
\[
	X_N\ni (r_{k,N}, \theta_{j,N}) = \left(\sqrt{\frac{1+\rho_{k,N}}{2}} , \frac{2\pi j}{4N} \right), \qquad (k=1,\ldots,N,j=1,\ldots,4N),
\]
and the respective weight values and constants are

\begin{gather}
	w(r_{k,N},\theta_{j,N}) = \frac{\mathcal{A}_{k,N}}{8N},\qquad \lambda_{n,m} = n+1,
	\label{weight}
\end{gather}
so the generated integral approximation based on \eqref{eq:QZMappr} is
\[
	\frac{1}{\pi} \int_{0}^1 \int_0^{2\pi} f(r,\theta)\ d\theta dr \approx \int_{X_N} f = \sum_{k=1}^{N} \sum_{j=1}^{4N} f(r_{k,N},\theta_{j,N}) \frac{\mathcal{A}_{k,N}}{8N}.
\]
We remark that the values $\mathcal{A}_{k,N}$ are somewhat close to $\frac{1}{N}$, so the weighting is close to being uniform.

What is appealing in this choice of $X_N$ is the fact that the orthogonality of quaternion Zernike functions \eqref{QZortho} is preserved under changing to  discrete integration over this set of points, i.e.
\begin{theorem}\label{QZdisc-ortho}
Suppose that for $n,n' \in \N$ naturals and $m,m' \in \Z$ integers we have $$\frac{n+n'}{2}+\min(|m|,|m'|) < 2N.$$ Then
\[
	(n+1) \int_{X_N} \phi_{n,m} \phi^*_{n',m'} =\delta_{n,n'}\delta_{m,m'}.
\]
\end{theorem}
\noindent\textbf{Proof.}
Let us consider the radial orthogonality relation \eqref{Rortho} expressed with the Jacobi polynomials of \eqref{RJacobi}, i.e.
\[
\begin{gathered}
	\frac{1}{2n+2} \delta_{n,n'} = \int_0^1 R_{n,|m|}(r) R_{n',|m|}(r)r\ dr  = \\
	\int_0^1 r^{2|m|} P_{\frac{n - |m|}{2}}^{(0,|m|)}(2r^2-1) P_{\frac{n' - |m|}{2}}^{(0,|m|)}(2r^2-1) r\ dr,
\end{gathered}
\]
and apply a change of variable $u=2r^2-1$, to obtain
\[
	\frac{1}{2n+2} \delta_{n,n'} = \frac{1}{4} \int_{-1}^1 \left(\frac{1+u}{2}\right)^{2|m|} P_{\frac{n - |m|}{2}}^{(0,|m|)}(u) P_{\frac{n' - |m|}{2}}^{(0,|m|)}(u)\ du.
\]
Notice that here the integrand is a polynomial of degree $\frac{n+n'}{2}+|m| < 2N$, so the exact value of the integral is equal to the value of the Gauss-Legendre quadrature (see e.g. \cite{Szego}), i.e.
\[
	\frac{1}{2n+2} \delta_{n,n'} = \frac{1}{4} \sum_{k=1}^{N} \mathcal{A}_{k,N} \left(\frac{1+\rho_{k,N}}{2}\right)^{2|m|} P_{\frac{n - |m|}{2}}^{(0,|m|)}(\rho_{k,N}) P_{\frac{n' - |m|}{2}}^{(0,|m|)}(\rho_{k,N}).
\]

Rewriting this in terms of $r_{k,N} = \sqrt{\frac{1+\rho_{k,N}}{2}}$ gives a discrete orthogonality relation for the radial polynomials, as
\begin{equation}\label{Rdisc-ortho}
	\frac{1}{2n+2} \delta_{n,n'} = \frac{1}{4} \sum_{k=1}^{N} \mathcal{A}_{k,N} R_{n,m}(r_{k,N}) R_{n',m}(r_{k,N}).
\end{equation}

Now we can proceed with proving the statement. 
\[
\begin{gathered}
	\int_{X_N} \phi_{n,m} \phi^*_{n',m'} = \sum_{k=1}^{N} \sum_{j=1}^{4N} \phi_{n,m}(r_{k,N},\theta_{j,N}) \phi^*_{n',m'}(r_{k,N},					\theta_{j,N}) \frac{\mathcal{A}_{k,N}}{8N} = \\
	\frac{1}{8N} \left( \sum_{j=1}^{4N} e^{-\qmu m\theta_{j,N}}e^{\qmu m' \theta_{j,N}} \right) \left( \sum_{k=1}^N \mathcal{A}_{k,N} R_{n,m}(r_{k,N}) R_{n',m'}(r_{k,N}) \right),
\end{gathered}
\]
and since quaternions $-\qmu m\theta_{j,N}$ and $\qmu m' \theta_{j,N}$ commute, we have
\[
	\sum_{j=1}^{4N} e^{-\qmu (m-m')\frac{2\pi j}{4N}} = 4N\delta_{m,m'}
\]
for the first sum, so if $m\neq m'$, the discrete integral equals to $0$ and we are done.

Suppose $m=m'$, now we are left with
\[
	\int_{X_N} \phi_{n,m} \phi^*_{n',m} = \frac{1}{2} \sum_{k=1}^N \mathcal{A}_{k,N} R_{n,m}(r_{k,N}) R_{n',m}(r_{k,N}) = \frac{1}{n+1}\delta_{n,n'},
\]
where we used \eqref{Rdisc-ortho} for the last equation as we have $\frac{n+n'}{2}+|m| < 2N$.
\qed

Theoretically, the discrete orthogonality relation of Theorem \ref{QZdisc-ortho} means that the exact values of right and left moments can be computed, provided we can measure the function values over $X_N$, for a sufficiently large $N$: let us consider an arbitrary linear combination of right moments \eqref{eq:QZRM} in the form
\[
	f \approx f_N(r,\theta) = \mathop{\sum\sum}_{n+m<2N} Z^R_{n,m}(f) \phi^*_{n,m} (r,\theta),
\]
then the exact value of the moment $Z^R_{n,m}(f)$, using the previous result, is
\[
	Z^R_{n,m}(f) = \int_{X_N} f_N\phi_{n,m},
\]
and the same can be said for the left moments. This proves that for system $X_N$, the discretization error is zero, a property that no other previously used quaternion moment method possessed. Besides this, only numerical roundoffs and the moment order threshold $f \approx f_N$ generates computational inaccuracies.