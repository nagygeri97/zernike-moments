Moment invariants are widely used in applications for pattern matching~\cite{app2}, image recognition~\cite{pattern_recognition}, or to extract useful features from images~\cite{zernike_nn}.

Most of the moments are defined for single-channel, grayscale images, however extending these techniques to multichannel color images is an important and generally unresolved problem with many possible applications. Conventionally, for color images either RGB decomposition or grayscale conversion was used in order to utilize the methods defined for grayscale images. This may lead to loss of information, for example in the case of grayscale conversion (where the average of the color channels is taken) some color information can be lost.

More recently, the algebra of quaternions has been used to extend the single-channel methods to color images. For example, quaternion Fourier-Mellin moments have been introduced as an extension of the conventional Fourier-Mellin moments~\cite{qfmm}, as well as the quaternion Zernike moments as an extension of the conventional Zernike moments~\cite{qzm}.

The Zernike functions are a system of orthogonal functions defined over the unit disk. Using these functions as a basis for series expansions proved to be useful because of certain inherent invariance properties. Zernike moments, and by extension quaternion Zernike moments are defined by these functions.

Considering a digital image as a discrete sampling of an image function defined over a continuous domain, the need arises to discretize the computation of these moments. One important property of the discretization of these methods defined by orthogonal functions is to preserve the orthogonality over the discrete system, so as to avoid redundancy and achieve high robustness with respect to noise.

The conventional method for discretizing quaternion Zernike moments (used by \citeauthor{qzmi}~\cite{qzmi}) consists of uniformly distributed points over the unit disk. This method does not achieve discrete orthogonality thus decreasing the robustness of the moments. 

Our goal was to create a system over which the quaternion extension of the Zernike functions is discrete orthogonal and thus improve the robustness of quaternion Zernike moments and to decrease the error introduced by discretization.


\section{Contributions}
In this thesis a new method is proposed for the discretization of quaternion Zernike moments over the unit disk. A points system is constructed on the unit disk, over which the Zernike functions extended to quaternions are discrete orthogonal. 

This new method is compared to the method used by \citeauthor{qzmi}~\cite{qzmi}. For the tests, image sets from the Columbia Object Image Library~\cite{coil} and the Amsterdam Library of Object Images~\cite{aloi} were used. 

Besides the theoretical invariance properties of QZMIs, these are also verified empirically. The image reconstruction capabilities of both methods are compared and we find that the proposed method decreases the error of reconstruction significantly.

The methods are also applied to the recognition of rotated, scaled and translated (RST transformed) images with varying levels of either Gaussian or salt-and-pepper noise. We find that with respect to Gaussian noise the new method achieves significantly better rates of recognition, even for images with high noise values. For salt-and-pepper noise no significant difference can be found between the capabilities of the methods.
Additionally, we also show that by decreasing the number of points used for discretization, the new method is able to achieve similar results as the original method with high number of points, but the computational need to obtain these results is much lower using the new method.


\section{Structure of the thesis}
This section serves as an overview of the structure of this thesis and contains a short summary of each chapter.

Chapter~\ref{sec:intro} serves as an introduction, where the motivation for this work is described and the contributions featured in this thesis are presented.

Chapter~\ref{sec:background} provides the background for our work. The core concepts, such as (quaternion) Zernike moments and invariants are described in detail. A summary of previous methods for the discretization of quaternion Zernike moments is also given. Finally, some applications relying on Zernike moments are described.

The method we propose for the discretization of Zernike moments is presented in Chapter~\ref{sec:maths}. The construction of a discrete orthogonal points system and the proof of discrete orthogonality is also given.

The different methods and technologies used for the computation of the invariants is shown in Chapter~\ref{sec:implementation}. The computation of the proposed new system is also described.

Chapter~\ref{sec:tests} contains the description of all the tests conducted on the methods, such as image reconstruction or recognition. The results of these tests are evaluated and the original and proposed methods are compared.

Finally, Chapter~\ref{sec:conclusion} summarizes the work and results presented in this thesis. Future development possibilities are also presented. 